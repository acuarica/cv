%%%%%%%%%%%%%%%%%%%%%%%%%%%%%%%%%%%%%%%%%
% Twenty Seconds Resume/CV
% LaTeX Template
% Version 1.1 (8/1/17)
%
% This template has been downloaded from:
% http://www.LaTeXTemplates.com
%
% Original author:
% Carmine Spagnuolo (cspagnuolo@unisa.it) with major modifications by 
% Vel (vel@LaTeXTemplates.com)
%
% License:
% The MIT License (see included LICENSE file)
%
%%%%%%%%%%%%%%%%%%%%%%%%%%%%%%%%%%%%%%%%%

%----------------------------------------------------------------------------------------
%	PACKAGES AND OTHER DOCUMENT CONFIGURATIONS
%----------------------------------------------------------------------------------------

\documentclass[letterpaper]{twentysecondcv} % a4paper for A4

%----------------------------------------------------------------------------------------
%	 PERSONAL INFORMATION
%----------------------------------------------------------------------------------------

% If you don't need one or more of the below, just remove the content leaving the command, e.g. \cvnumberphone{}

\profilepic{alice.jpeg} % Profile picture

\cvname{Luis Mastrangelo} % Your name
\cvjobtitle{Adventurer} % Job title/career

\cvdate{31 January 1981} % Date of birth
\cvaddress{Via Zurigo 1\newline 6900 Lugano, Switzerland} % Short address/location, use \newline if more than 1 line is required
\cvnumberphone{+41 789 44 2022} % Phone number
\cvsite{https://acuarica.gitlab.io} % Personal website
\cvmail{luis.mastrangelo@usi.ch} % Email address

\begin{document}

\aboutme{Alice is a sensible prepubescent girl from a wealthy English family who finds herself in a strange world ruled by imagination and fantasy. Alice feels comfortable with her identity and has a strong sense that her environment is comprised of clear, logical, and consistent rules and features. Alice's familiarity with the world has led one critic to describe her as a "disembodied intellect". Alice displays great curiosity and attempts to fit her diverse experiences into a clear understanding of the world.} % To have no About Me section, just remove all the text and leave \aboutme{}

% Skill bar section, each skill must have a value between 0 an 6 (float)
\skills{{pursuer of rabbits/5.8},{good manners/4},{outgoing/4.3},{polite/4},{Java/0.01}}

%------------------------------------------------

% Skill text section, each skill must have a value between 0 an 6
\skillstext{{lovely/4},{narcissistic/3}}


\makeprofile % Print the sidebar

%----------------------------------------------------------------------------------------
%	 INTERESTS
%----------------------------------------------------------------------------------------

\section{Interests}

The heroine and the dreamer of Wonderland; Alice is the principal character.

%----------------------------------------------------------------------------------------
%	 EDUCATION
%----------------------------------------------------------------------------------------

\section{Education}

\begin{twenty} % Environment for a list with descriptions
	\twentyitem{since 1865}{Ph.D. {\normalfont candidate in Computer Science}}{Wonderland}{\emph{A Quantified Theory of Social Cohesion.}}
	\twentyitem{1863-1865}{M.Sc. magna cum laude}{Wonderland}{Majoring in Computer Science}
	\twentyitem{1861-1863}{B.Sc. magna cum laude}{Wonderland}{Majoring in Computer Science}
	\twentyitem{1856-1861}{High school}{Wonderland}{Specializing in mathematics and physics.}
	%\twentyitem{<dates>}{<title>}{<location>}{<description>}
\end{twenty}

%----------------------------------------------------------------------------------------
%	 PUBLICATIONS
%----------------------------------------------------------------------------------------

\section{Publications}

\begin{twentyshort} % Environment for a short list with no descriptions
	\twentyitemshort{1865}{Chapter One, Down the Rabbit Hole.}
	\twentyitemshort{1865}{Chapter Two, The Pool of Tears.}
	\twentyitemshort{1865}{Chapter Three,  The Caucus Race and a Long Tale.}
	\twentyitemshort{1865}{Chapter Four,  The Rabbit Sends a Little Bill.}
	\twentyitemshort{1865}{Chapter Five,  Advice from a Caterpillar.}
	%\twentyitemshort{<dates>}{<title/description>}
\end{twentyshort}

%----------------------------------------------------------------------------------------
%	 AWARDS
%----------------------------------------------------------------------------------------

\section{Awards}

\begin{twentyshort} % Environment for a short list with no descriptions
	\twentyitemshort{1987}{All-Time Best Fantasy Novel.}
	\twentyitemshort{1998}{All-Time Best Fantasy Novel before 1990.}
	%\twentyitemshort{<dates>}{<title/description>}
\end{twentyshort}

%----------------------------------------------------------------------------------------
%	 EXPERIENCE
%----------------------------------------------------------------------------------------

\section{Experience}

\begin{twenty} % Environment for a list with descriptions
	\twentyitem{1900}{Alice in Wonderland-The Circra (1900's) Silent Film.}{Film}{The first Alice on film was over a hundred years ago.}
	\twentyitem{1933}{Alice in Wonderland 1933 version.}{Film}{This film stars Ethel griffies and Charlotte Henry. It was a box office flop when it was released.}
	\twentyitem{1951}{Disney Film.}{Film}{Walt Disney brings Lewis Carroll's fantasy story to life in this well done animated classic. Even though many elements from the book were dropped, such as the duchess with the baby pig and mock turtle, this version is without a doubt the most famous Alice adaption made.}
	%\twentyitem{<dates>}{<title>}{<location>}{<description>}
\end{twenty}

%----------------------------------------------------------------------------------------
%	 OTHER INFORMATION
%----------------------------------------------------------------------------------------

\section{Other information}

\subsection{Review}

Alice approaches Wonderland as an anthropologist, but maintains a strong sense of noblesse oblige that comes with her class status. She has confidence in her social position, education, and the Victorian virtue of good manners. Alice has a feeling of entitlement, particularly when comparing herself to Mabel, whom she declares has a ``poky little house," and no toys. Additionally, she flaunts her limited information base with anyone who will listen and becomes increasingly obsessed with the importance of good manners as she deals with the rude creatures of Wonderland. Alice maintains a superior attitude and behaves with solicitous indulgence toward those she believes are less privileged.

%----------------------------------------------------------------------------------------
%	 SECOND PAGE EXAMPLE
%----------------------------------------------------------------------------------------

%\newpage % Start a new page

%\makeprofile % Print the sidebar

%\section{Other information}

%\subsection{Review}

%Alice approaches Wonderland as an anthropologist, but maintains a strong sense of noblesse oblige that comes with her class status. She has confidence in her social position, education, and the Victorian virtue of good manners. Alice has a feeling of entitlement, particularly when comparing herself to Mabel, whom she declares has a ``poky little house," and no toys. Additionally, she flaunts her limited information base with anyone who will listen and becomes increasingly obsessed with the importance of good manners as she deals with the rude creatures of Wonderland. Alice maintains a superior attitude and behaves with solicitous indulgence toward those she believes are less privileged.

%\section{Other information}

%\subsection{Review}

%Alice approaches Wonderland as an anthropologist, but maintains a strong sense of noblesse oblige that comes with her class status. She has confidence in her social position, education, and the Victorian virtue of good manners. Alice has a feeling of entitlement, particularly when comparing herself to Mabel, whom she declares has a ``poky little house," and no toys. Additionally, she flaunts her limited information base with anyone who will listen and becomes increasingly obsessed with the importance of good manners as she deals with the rude creatures of Wonderland. Alice maintains a superior attitude and behaves with solicitous indulgence toward those she believes are less privileged.

%----------------------------------------------------------------------------------------

\end{document} 






% % Possible document class options include:
% %  - Font size: 10pt | 11pt | 12pt
% %  - Paper size: a4paper | letterpaper | a5paper | legalpaper | executivepaper |  landscape
% %  - Font family: sans | roman
% \documentclass[11pt,a4paper,sans]{moderncv}

% % style options are 'casual' (default), 'classic', 'oldstyle' and 'banking'
% \moderncvstyle{classic}
% % color options 'blue' (default), 'orange', 'green', 'red', 'purple', 'grey' and 'black'
% \moderncvcolor{blue}

% %\usepackage{fancyhdr}
% %\pagestyle{fancy}
% %\fancyhead[L]{Curriculum Vitae}
% %\fancyhead[R]{Luis A. Mastrangelo}

% \usepackage[spanish, activeacute]{babel}
% \usepackage[scale=0.77]{geometry}

% \setlength{\hintscolumnwidth}{3.35cm}

% \firstname{Luis A.}
% \familyname{Mastrangelo}
% \title{Curriculum Vitae}
% \address{Via Zurigo 1}{6900 Lugano, Switzerland}
% \mobile{+41 789 44 2022}
% \email{luis.mastrangelo@usi.ch}
% % \homepage{acuarica.gitlab.io/blog/}

% \begin{document}

% \makecvtitle

% \section{Personal Details}

% \cvitem{Date of birth}{31/01/1981 \ \ \ \ \ \ \ \ \ Place of birth \ Ciudad Aut'onoma de Buenos Aires, Argentina}
% % \cvitem{Place of birth}{Ciudad Aut'onoma de Buenos Aires, Argentina}
% \cvitem{Nationality}{Argentine \ \ \ \ \ \ \ \ \ Marital status \ Single }
% % \cvitem{Marital status}{Single}

% \section{Research Experience}

% \cventry{Apr 2013--June 2019}{Ph.D. Graduate, Computer Science}{University of Lugano}{Switzerland}{}{
% Subject: programming languages, software verification and declarative programming. \\
% 	Advisors: Prof. Matthias Hauswirth and Prof. Nathaniel Nystrom \\
% 	Ph.D. Thesis: \emph{``When and How Java Developers Give Up Static Type Safety''}
% }
% \cventry{Jul 2012--Nov 2012}{Ph.D. Student, Computer Science}{University of Luxembourg}{Luxembourg}{}{Subject: automatic model based testing with focus on embedded systems. \\
% Advisor: Prof. Lionel Briand
% }
% \cventry{Apr 2011--Oct 2011}{Internship}{INRIA/University of Strasbourg}{Strasbourg, France}{}{Internship title: ``A Virtual Machine for Automatic Program Parallelization''.}

% \section{Professional Experience}
% \cventry{Current position \\ Aug 2019--\emph{to date}}{Technical Consultant}{Avaloq Banking Group AG}{Lugano, Switzerland}{}{IT infrastructure support for Monitoring Operations}
% \cventry{Dec 2006--Mar 2011}{Project Leader}{Recursiva S.R.L.}{Buenos Aires, Argentina}{}{With roles of software engineering and architect for ABB and Bayer among others.}
% \cventry{Dec 2010--Jan 2011}{Freelance project}{vCelebrate}{Buenos Aires, Argentina}{}{Technologies used: Facebook API and Adobe Flex.}
% \cventry{Apr--Dec 2008}{Freelance Project}{Andrea Importadora}{Buenos Aires, Argentina}{}{Technologies used: ASP.NET and MySQL.}
% \cventry{Mar--Jul 2006}{Network Administrator}{HS Eventos}{Buenos Aires, Argentina}{}{Technical service, software installation and maintenance of company network.}
% \cventry{Feb--Apr 2006}{Freelance Project}{Reid Systems, ApareSer}{Buenos Aires, Argentina}{}{Technologies used: ASP.NET, C\# and SQL Server 2000.}
% \cventry{Apr--Oct 2005}{Freelance Project}{Dream On}{Buenos Aires, Argentina}{}{Technologies used: C\# and PostgreSQL.}
% \cventry{Jul--Oct 2004}{Network Administrator}{Texpampa}{Buenos Aires, Argentina}{}{Technical service, software installation and maintenance of company network.}
% \cventry{Mar 2000--May 2004}{Software Developer}{Fernando Hermo y Asoc.}{Buenos Aires, Argentina}{}{Using SQL Server and Crystal Reports and technical support to customers.}

% \section{Educational Background}
% \cventry{Apr 2007--Nov 2011}{Licentiate in Computer Science}{Faculty of Exact and Natural Sciences, University of Buenos Aires}{Buenos Aires, Argentina}{}{
% The degree requires 23 mandatory courses,
% 12 points obtained by talking elective courses, and a final thesis.
% It is designated to be completed in 6 years.
% As such, it is considered by many institutions as being equivalent to an M.Sc. degree.
% From the 23 mandatory courses,
% 13 must be passed in order to obtain the intermediate \emph{Computer Analyst} degree.
% This degree is a 3 year degree.\\
% Advisor: Prof. Sergio Yovine, Prof. Philippe Clauss\\
% Licentiate Thesis: ``Dynamic Analysis and Transformation of Programs for Automatic and Speculative Parallelization: Contributions and Perspectives''.
% }

% \cventry{Apr 2003--Dec 2006}{Computer Analyst}{Faculty of Exact and Natural Sciences, University of Buenos Aires}{Buenos Aires, Argentina}{}{}

% \cventry{Mar 1994--Nov 1999}{Secondary School Diploma (Specialization in informatics)}{Technical School No. 3 Mar'ia S'anchez de Thompson}{Buenos Aires, Argentina}{}{}


% \section{Teaching Experience}
% \cvitem{Apr 2013--Nov 2017}{\textbf{Teaching Assistant}, \emph{University of Lugano}, Switzerland. Multiple courses.}
% \cvitem{Apr 2006--Nov 2010}{\textbf{Teaching Assistant}, \emph{University of Buenos Aires}, Argentina. Multiple courses.}
% \cvitem{Apr--Oct 2008}{\textbf{Scientific Popularizer}, \emph{Informatics dpt., University of Buenos Aires}, Argentina.}

% \section{Academic Services}
% \cvitem{Jun--Aug 2016}{\textbf{Program Committee}, OOPSLA'16 Artifact Evaluation Track}

% \renewcommand{\refname}{Publications}
% \nocite{*}
% \bibliographystyle{unsrt}
% \bibliography{pubs}

% \section{Language Proficiency}
% \cvitem{Spanish}{\textbf{Native speaker}}
% \cvitem{English}{\textbf{Proficient User}: Level C2 under CEFR}
% \cvitem{Italian}{\textbf{Proficient User}: Level C1 under CEFR}
% \cvitem{French}{\textbf{Independent User}: Level B1 under CEFR}
% \cvitem{German}{\textbf{Basic User}: Level A1 under CEFR}

% \end{document}
