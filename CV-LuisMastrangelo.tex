\documentclass[11pt,a4paper,sans]{moderncv}

\moderncvstyle{classic}
\moderncvcolor{blue}

\usepackage[spanish, activeacute]{babel}
\usepackage[scale=0.77]{geometry}

\setlength{\hintscolumnwidth}{3.35cm}

\photo[79pt]{avatar}
\firstname{Luis}
\familyname{Mastrangelo}
\title{Software Development Engineer}
\address{Via Zurigo 1}{6900 Lugano, Switzerland}
\mobile{+41 789 44 2022}
\email{luismastrangelo@gmail.com}
\homepage{acuarica.gitlab.io/blog/}

\begin{document}

\makecvtitle

\section{Education}
\cventry{Apr 2013--June 2019}{Ph.D. Graduate, Computer Science}{University of Lugano}{Switzerland}{}{
Programming languages research in Java and Haskell, runtime systems and virtual machines with C++, software verification, data analysis using R and Python.}
\cventry{Apr 2007--Nov 2011}{Licentiate in Computer Science}{University of Buenos Aires}{Argentina}{}{
This degree is considered by many institutions as being equivalent to an M.Sc. degree.}
\cventry{Apr 2011--Oct 2011}{Internship}{INRIA/University of Strasbourg}{France}{}{Internship title: ``A Virtual Machine for Automatic Program Parallelization''.}
\cventry{Apr 2003--Dec 2006}{Computer Analyst}{University of Buenos Aires}{Argentina}{}{
This is a 3 years degree as part of the \emph{Licentiate in Computer Science} programme.}
\cventry{Mar 1994--Nov 1999}{Secondary School Diploma in Informatics}{}{Buenos Aires, Argentina}{}{}

\section{Professional Experience}
\cventry{Current position \\ Aug 2019--\emph{to date}}{Technical Consultant}{DB\&L Services SA}{Lugano, Switzerland}{}{IT infrastructure support for Monitoring Operations at Avaloq Sourcing (CH \& LI) SA.}
\cventry{Dec 2006--Mar 2011}{Project Leader}{Recursiva S.R.L.}{Buenos Aires, Argentina}{}{Software engineering and architect roles using Java and C\# among others technologies.}
\cventry{Mar--Jul 2006}{Network Administrator}{HS Eventos}{Buenos Aires, Argentina}{}{Technical service, software installation and maintenance of company network.}
\cventry{Jul--Oct 2004}{Network Administrator}{Texpampa}{Buenos Aires, Argentina}{}{Technical service, software installation and maintenance of company network.}
\cventry{Mar 2000--May 2004}{Software Developer}{Fernando Hermo y Asoc.}{Buenos Aires, Argentina}{}{ERP and CRM development using VB 6, SQL Server and Crystal Reports.}

\section{Freelance \& Open Source Projects}
\cventry{Feb 2014--Jan 2020}{Software Developer}{\href{https://gitlab.com/acuarica/jnif}{JNIF}}{Lugano, Switzerland}{}{Java Native Instrumentation Framework written in C++.}
\cventry{May--Sep 2018}{Software Developer}{Essentia, Family Office}{Lugano, Switzerland}{}{Desktop application written in Scala.}
\cventry{May--Jun 2018}{Software Engineer}{GTM Open Source project (contribution)}{Remote}{}{Add support for submodules in Git Time Tracking. Written in Go. \href{https://github.com/git-time-metric/gtm/pull/85}{Pull Request}.}
\cventry{Oct--Nov 2017}{Software Developer}{Essentia, Family Office}{Lugano, Switzerland}{}{Desktop application written in C\#.}
\cventry{Apr 2016--Oct 2017}{Software Developer}{\href{https://gitlab.com/acuarica/hsc}{HSC}}{Remote}{}{A Supercompiler prototype for Haskell.}
\cventry{Dec 2010--Jan 2011}{Web Developer}{vCelebrate}{Buenos Aires, Argentina}{}{Web application using Facebook API and Adobe Flex.}
\cventry{Apr--Dec 2008}{Full Stack Developer}{Andrea Importadora}{Buenos Aires, Argentina}{}{E-commerce platform using ASP.NET and MySQL.}
\cventry{Feb--Apr 2006}{Full Stack Developer}{Reid Systems, ApareSer}{Buenos Aires, Argentina}{}{Web application using ASP.NET, C\# and SQL Server 2000.}
\cventry{Apr--Oct 2005}{Full Stack Developer}{Dream On}{Buenos Aires, Argentina}{}{Invoicing desktop application using C\# and PostgreSQL.}

\section{Technical Skills}
\newcommand{\slashsep}{{\footnotesize /}}
\cvitem{Languages}{Java, Kotlin, Scala, C\slashsep{}C{\scriptsize ++}, C{\scriptsize \#}, Haskell, Python, R, JavaScript\slashsep{}TypeScript, Go, SQL, Bash, PHP, \LaTeX, UML, LLVM, Perl, VB.NET\slashsep{}6, Assembler, Pascal}
\cvitem{Paradigms}{OOP, Functional \& Logic Programming, Design by Contract, Design Patterns}
\cvitem{Fields}{Static \& Dynamic Program Analysis, Data Science, Mining, Visualization}
\cvitem{Databases}{Design, Oracle PL\slashsep{}SQL, Datalog\slashsep{}QL, PostgreSQL, MySQL, SQL Server, SQLite}
\cvitem{Editors/IDEs}{VSCode, IntelliJ\slashsep{}CLion\slashsep{}Rider, Visual Studio.NET, Eclipse, Xcode, Vim\slashsep{}Emacs}
\cvitem{SCMs \& IT}{Git, Subversion, CVS, Linux, Scripting, Virtualization, Samba, Apache, Tomcat}

\section{Teaching Experience}
\cvitem{Apr 2013--Nov 2017}{\textbf{Teaching Assistant}, \emph{University of Lugano}, Switzerland. Multiple courses.}
\cvitem{Apr 2006--Nov 2010}{\textbf{Teaching Assistant}, \emph{University of Buenos Aires}, Argentina. Multiple courses.}
\cvitem{Apr--Oct 2008}{\textbf{Scientific Popularizer}, \emph{Informatics dpt., University of Buenos Aires}, Argentina.}

\section{Publications}
\cvitem{University of Lugano}{\textbf{Luis Mastrangelo}, Ph.D. Thesis, Advisors: Prof. M. Hauswirth and Prof. N. Nystrom. \href{https://doc.rero.ch/record/327253}{When and How Java Developers Give Up Static Type Safety}. 2019}
\cvitem{University of Lugano}{\textbf{Luis Mastrangelo}, M. Hauswirth and N. Nystrom. \href{https://doi.org/10.1145/3360584}{Casting about in the Dark: A Survey of Cast Usage in Java}. 2019}
\cvitem{University of Lugano}{\textbf{Luis Mastrangelo}, L. Ponzanelli, A. Mocci, M. Lanza, M. Hauswirth and N. Nystrom. \href{https://doi.org/10.1145/2858965.2814313}{Use at Your Own Risk: The Java Unsafe API in the Wild}. 2015}
\cvitem{University of Lugano}{\textbf{Luis Mastrangelo} and M. Hauswirth. \href{https://doi.org/10.1145/2647508.2647516}{JNIF: Java Native Instrumentation Framework}. 2014}
\cvitem{INRIA/University of Strasbourg}{A. Jimborean, \textbf{Luis Mastrangelo}, V. Loechner and P. Clauss. \href{https://doi.org/10.1007/978-3-642-28652-0_12}{VMAD: an Advanced Dynamic Program Analysis \& Instrumentation Framework}. 2012}
\cvitem{INRIA/University of Strasbourg}{A. Jimborean, P. Clauss, B. Pradelle, \textbf{Luis Mastrangelo} and V. Loechner. \href{https://doi.org/10.1145/2370036.2145861}{Adapting the Polyhedral Model for Efficient Speculative Parallelization}. 2012}

\section{Language Proficiency}
\newcommand{\CEFR}{\href{https://www.coe.int/web/common-european-framework-reference-languages/table-1-cefr-3.3-common-reference-levels-global-scale}{CEFR}}
\cvitem{Spanish}{\textbf{Native speaker}}
\cvitem{English}{\textbf{Proficient User}: Level C2 under \CEFR{}}
\cvitem{Italian}{\textbf{Proficient User}: Level C1 under \CEFR{}}
\cvitem{French}{\textbf{Independent User}: Level B1 under \CEFR{}}

\end{document}